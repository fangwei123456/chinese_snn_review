% !TeX encoding = UTF-8
% !TeX program = xelatex
% !TeX spellcheck = <none>

%-----------------------------------------------------------------------
% 中国科学: 信息科学 中文模板, 请用 CCT-LaTeX 编译
% http://scis.scichina.com
% 南开大学程明明注释:也可以在Overleaf中使用XeLaTeX直接编译,
% 例如:
%-----------------------------------------------------------------------

\documentclass{SCIS2020cn}
%\usepackage{breakurl}
%\captionsetup[subfloat]{labelformat=simple,captionskip=0pt}


%%%%%%%%%%%%%%%%%%%%%%%%%%%%%%%%%%%%%%%%%%%%%%%%%%%%%%%
%%% 作者附加的定义
%%% 常用环境已经加载好, 不需要重复加载
%%%%%%%%%%%%%%%%%%%%%%%%%%%%%%%%%%%%%%%%%%%%%%%%%%%%%%%


%%%%%%%%%%%%%%%%%%%%%%%%%%%%%%%%%%%%%%%%%%%%%%%%%%%%%%%
%%% 开始
%%%%%%%%%%%%%%%%%%%%%%%%%%%%%%%%%%%%%%%%%%%%%%%%%%%%%%%
\begin{document}

%%%%%%%%%%%%%%%%%%%%%%%%%%%%%%%%%%%%%%%%%%%%%%%%%%%%%%%
%%% 作者不需要修改此处信息
\ArticleType{论文}
%\SpecialTopic{}
%\Luntan{中国科学院学部\quad 科学与技术前沿论坛}
\Year{2020}
\Vol{50}
\No{1}
\BeginPage{1}
\DOI{}
\ReceiveDate{}
\ReviseDate{}
\AcceptDate{}
\OnlineDate{}
%%%%%%%%%%%%%%%%%%%%%%%%%%%%%%%%%%%%%%%%%%%%%%%%%%%%%%%

\title{深度脉冲神经网络梯度替代学习算法研究综述}{引用的标题}

\entitle{Title}{Title for citation}

\author[1,2]{a}{}
\author[2]{b}{{abc@xxxx.xxx}}

\enauthor[1,2]{Ming XING}{}
\enauthor[2]{Mingming XING}{{abc@xxxx.xxx}}
\enauthor[1]{Ming XING}{}
\enauthor[3]{Ming XING}{}

\address[1]{作者单位, 城市 000000}
\address[2]{作者单位, 城市 000000}
\address[3]{作者单位, 城市 000000}

\enaddress[1]{Affiliation, City {\rm 000000}, Country}
\enaddress[2]{Affiliation, City {\rm 000000}, Country}
\enaddress[3]{Affiliation, City {\rm 000000}, Country}

\Foundation{基金资助}

\AuthorMark{第一作者等}

\AuthorCitation{作者1, 作者2, 作者3, 等}
\enAuthorCitation{Xing M, Xing M M, Xing M, et al}

%\comment{\dag~同等贡献}
%\encomment{\dag~Equal contribution}

\abstract{摘要主要包括本文的研究目的、方法、结果和结论, 注意突出创新点. 应避免出现图、表、公式、参考文献引用等. 对应的英文摘要长度在200词左右.}

\enabstract{An abstract (about 200 words) is a summary of the content of the manuscript. It should briefly describe the research purpose, method, result and conclusion. The extremely professional terms, special signals, figures, tables, chemical structural formula, and equations should be avoided here, and citation of references is not allowed.}

\keywords{关键词1, 关键词2, 关键词3, 关键词4, 关键词5}

\enkeywords{keyword1, keyword2, keyword3, keyword4, keyword5}

\maketitle

% 小上标引用 \upcite

% 图片示例
%\begin{figure}[!t]
%\centering
%%\includegraphics{fig1.eps}
%\cnenfigcaption{图题}{Caption}
%\label{fig1}
%\end{figure}

% 公式示例
%\begin{eqnarray}
%	\nonumber
%	X&=&[x_{11},x_{12},\ldots,x_{ij},\ldots ,x_{n-1,n}]^{\rm T},\\
%	\nonumber
%	\varepsilon&=&[e_{11},e_{12},\ldots ,e_{ij},\ldots ,e_{n-1,n}],\\
%	\nonumber
%	T&=&[t_{11},t_{12},\ldots ,t_{ij},\ldots ,t_{n-1,n}].
%\end{eqnarray}

% 表格示例

%\begin{table}[!t]
%	\cnentablecaption{表题}{Caption}
%	\label{tab1}
%	\footnotesize
%	\tabcolsep 40pt %space between two columns. 用于调整列间距
%	\begin{tabular*}{\textwidth}{cccc}
%		\toprule
%		Title a & Title b & Title c & Title d \\\hline
%		Aaa & Bbb & Ccc & Ddd\\
%		Aaa & Bbb & Ccc & Ddd\\
%		Aaa & Bbb & Ccc & Ddd\\
%		\bottomrule
%	\end{tabular*}
%\end{table}

% 算法示例

%\begin{algorithm}
%	%\floatname{algorithm}{Algorithm}%更改算法前缀名称
%	\renewcommand{\algorithmicrequire}{\textbf{输入:}}% 更改输入名称
%	\renewcommand{\algorithmicensure}{\textbf{主迭代:}}% 更改输出名称
%	\newcommand{\LASTCON}{\item[\algorithmiclastcon]}
%	\newcommand{\algorithmiclastcon}{\textbf{输出:}}% 更改输出名称
%	\footnotesize
%	\caption{算法标题}
%	\label{alg1}
%	\begin{algorithmic}[1]
%		\REQUIRE $n \geq 0 \vee x \neq 0$;
%		\ENSURE $y = x^n$;
%		\STATE $y \Leftarrow 1$;
%		\IF{$n < 0$}
%		\STATE $X \Leftarrow 1 / x$;
%		\STATE $N \Leftarrow -n$;
%		\ELSE
%		\STATE $X \Leftarrow x$;
%		\STATE $N \Leftarrow n$;
%		\ENDIF
%		\WHILE{$N \neq 0$}
%		\IF{$N$ is even}
%		\STATE $X \Leftarrow X \times X$;
%		\STATE $N \Leftarrow N / 2$;
%		\ELSE[$N$ is odd]
%		\STATE $y \Leftarrow y \times X$;
%		\STATE $N \Leftarrow N - 1$;
%		\ENDIF
%		\ENDWHILE
%		\LASTCON
%	\end{algorithmic}
%\end{algorithm}



\section{引言}

\section{脉冲神经网络的常用概念和评测基准}

\section{脉冲神经网络的梯度替代训练算法}

\subsection{基础学习算法}% 例如slayer、stbp

SLAYER: Spike Layer Error Reassignment in Time

Spatio-temporal backpropagation for training high-performance spiking neural networks

SuperSpike: Supervised learning in multi-layer spiking neural networks

Differentiable spike: Rethinking gradient-descent for training spiking neural networks

\subsection{ANN辅助训练}

A Tandem Learning Rule for Effective Training and Rapid Inference of Deep Spiking Neural Networks

Distilling Spikes: Knowledge Distillation in Spiking Neural Networks

Constructing Deep Spiking Neural Networks from Artificial Neural Networks with Knowledge Distillation

Self-Architectural Knowledge Distillation for Spiking Neural Networks


\subsection{神经元和突触改进}% 例如plif、glif

Incorporating learnable membrane time constant to enhance learning of spiking neural networks

GLIF: A unified gated leaky integrate-and-fire neuron for spiking neural networks

Multi-level firing with spiking ds-resnet: Enabling better and deeper directly-trained spiking neural networks

Parallel Spiking Neurons with High Efficiency and Ability to Learn Long-term Dependencies

Temporal backpropagation for spiking neural networks with one spike per neuron

CLIF: Complementary Leaky Integrate-and-Fire Neuron for Spiking Neural Networks


Learning Delays in Spiking Neural Networks using Dilated Convolutions with Learnable Spacings

Exploiting Neuron and Synapse Filter Dynamics in Spatial Temporal Learning of Deep Spiking Neural Network


\subsection{网络结构改进}% 例如sew resnet、spikformer、注意力机制

Spiking deep residual network

Deep residual learning in spiking neural networks

Advancing Spiking Neural Networks towards Deep Residual Learning

Temporal-wise Attention Spiking Neural Networks for Event Streams Classification

Attention Spiking Neural Networks

Inherent Redundancy in Spiking Neural Networks

Spike-based dynamic computing with asynchronous sensing-computing neuromorphic chip

Spikformer: When spiking neural network meets transformer

SpikingResformer: Bridging ResNet and Vision Transformer in Spiking Neural Networks

Spike-driven Transformer

Spike-driven Transformer V2: Meta Spiking Neural Network Architecture Inspiring the Design of Next-generation Neuromorphic Chips

QKFormer: Hierarchical Spiking Transformer using Q-K Attention

AutoSNN: Towards Energy-Efficient Spiking Neural Networks

Neural Architecture Search for Spiking Neural Networks

Differentiable hierarchical and surrogate gradient
search for spiking neural networks

\subsection{正则化方法}% 例如tdbn、tebn

Direct training for spiking neural networks: Faster, larger, better

Going deeper with directly-trained larger spiking neural networks

Neuromorphic Data Augmentation for Training Spiking Neural Networks

Revisiting Batch Normalization for Training Low-Latency Deep Spiking Neural Networks From Scratch

Temporal Effective Batch Normalization in Spiking Neural Networks

Temporal efficient training of spiking neural network via gradient re-weighting

RMP-Loss: Regularizing Membrane Potential Distribution for Spiking Neural Networks

Membrane Potential Batch Normalization for Spiking Neural Networks


\subsection{事件驱动学习算法}% 朱耀宇的工作

Hybrid macro/micro level backpropagation for training deep spiking neural networks

Spike-train level backpropagation for training deep recurrent spiking neural networks

Temporal spike sequence learning via backpropagation for deep spiking neural networks

Training spiking neural networks with event-driven backpropagation

Exploring Loss Functions for Time-based Training Strategy in Spiking Neural Networks

\subsection{在线学习算法}% 朱耀宇、孟庆晏的工作

Synaptic plasticity dynamics for deep continuous local learning (decolle)

Online training through time for spiking neural networks

Towards memory-and time-efficient backpropagation for training spiking neural networks

Online stabilization of spiking neural networks

High-Performance Temporal Reversible Spiking Neural Networks with O(L) Training Memory and O(1) Inference Cost

NDOT: Neuronal Dynamics-based Online Training for Spiking Neural Networks


\subsection{训练加速方法}
Sparse spiking gradient descent

SpikingJelly: An open-source machine learning infrastructure platform for spike-based intelligence

Addressing the speed-accuracy simulation trade-off for adaptive spiking neurons

%%%%%%%%%%%%%%%%%%%%%%%%%%%%%%%%%%%%%%%%%%%%%%%%%%%%%%%
%%% 致谢
%%% 非必选
%%%%%%%%%%%%%%%%%%%%%%%%%%%%%%%%%%%%%%%%%%%%%%%%%%%%%%%
%\Acknowledgements{致谢.}

%%%%%%%%%%%%%%%%%%%%%%%%%%%%%%%%%%%%%%%%%%%%%%%%%%%%%%%
%%% 补充材料说明
%%% 非必选
%%%%%%%%%%%%%%%%%%%%%%%%%%%%%%%%%%%%%%%%%%%%%%%%%%%%%%%
%\Supplements{补充材料.}

%%%%%%%%%%%%%%%%%%%%%%%%%%%%%%%%%%%%%%%%%%%%%%%%%%%%%%%
%%% 参考文献, {}为引用的标签, 数字/字母均可
%%% 文中上标引用: \upcite{1,2}
%%% 文中正常引用: \cite{1,2}
%%%%%%%%%%%%%%%%%%%%%%%%%%%%%%%%%%%%%%%%%%%%%%%%%%%%%%%


%\newpage
\bibliographystyle{plain}
\bibliography{ref}

%%%%%%%%%%%%%%%%%%%%%%%%%%%%%%%%%%%%%%%%%%%%%%%%%%%%%%%
%%% 附录章节, 自动从A编号, 以\section开始一节
%%% 非必选
%%%%%%%%%%%%%%%%%%%%%%%%%%%%%%%%%%%%%%%%%%%%%%%%%%%%%%%
%\begin{appendix}
%\section{附录}
%附录从这里开始.
%\begin{figure}[H]
%\centering
%%\includegraphics{fig1.eps}
%\cnenfigcaption{附录里的图}{Caption}
%\label{fig1}
%\end{figure}
%\end{appendix}


%%%%%%%%%%%%%%%%%%%%%%%%%%%%%%%%%%%%%%%%%%%%%%%%%%%%%%%
%%% 自动生成英文标题部分
%%%%%%%%%%%%%%%%%%%%%%%%%%%%%%%%%%%%%%%%%%%%%%%%%%%%%%%
\makeentitle


%%%%%%%%%%%%%%%%%%%%%%%%%%%%%%%%%%%%%%%%%%%%%%%%%%%%%%%
%%% 主要作者英文简介, 数量不超过4个
%%% \authorcv[zp1.eps]{Ming XING}{was born in ...}
%%% [照片文件名]请提供清晰的一寸浅色背景照片, 宽高比为 25:35
%%% {姓名}与英文标题处一致
%%%%%%%%%%%%%%%%%%%%%%%%%%%%%%%%%%%%%%%%%%%%%%%%%%%%%%%
\authorcv[]{Ming XING}{was born in ...}

\authorcv[]{Ming XING}{was born in ...}

%\vspace*{6mm} % 调整照片行间距

\authorcv[]{Ming XING}{was born in ...}

\authorcv[]{Ming XING}{was born in ...}



%%%%%%%%%%%%%%%%%%%%%%%%%%%%%%%%%%%%%%%%%%%%%%%%%%%%%%%
%%% 补充材料, 以附件形式作网络在线, 不出现在印刷版中
%%% 不做加工和排版, 仅用于获得图片和表格编号
%%% 自动从I编号, 以\section开始一节
%%% 可以没有\section
%%%%%%%%%%%%%%%%%%%%%%%%%%%%%%%%%%%%%%%%%%%%%%%%%%%%%%%
%\begin{supplement}
%\section{supplement1}
%自动从I编号, 以section开始一节.
%\begin{figure}[H]
%\centering
%\includegraphics{fig1.eps}
%\cnenfigcaption{补充材料里的图}{Caption}
%\label{fig1}
%\end{figure}
%\end{supplement}

\end{document}


%%%%%%%%%%%%%%%%%%%%%%%%%%%%%%%%%%%%%%%%%%%%%%%%%%%%%%%
%%% 本模板使用的latex排版示例
%%%%%%%%%%%%%%%%%%%%%%%%%%%%%%%%%%%%%%%%%%%%%%%%%%%%%%%

%%% 章节
\section{}
\subsection{}
\subsubsection{}


%%% 普通列表
\begin{itemize}
\item Aaa aaa.
\item Bbb bbb.
\item Ccc ccc.
\end{itemize}

%%% 自由编号列表
\begin{itemize}
\itemindent 4em
\item[(1)] Aaa aaa.
\item[(2)] Bbb bbb.
\item[(3)] Ccc ccc.
\end{itemize}

%%% 定义、定理、引理、推论等, 可用下列标签
%%% definition 定义
%%% theorem 定理
%%% lemma 引理
%%% corollary 推论
%%% axiom 公理
%%% propsition 命题
%%% example 例
%%% exercise 习题
%%% solution 解名
%%% notation 注
%%% assumption 假设
%%% remark 注释
%%% property 性质
%%% []中的名称可以省略, \label{引用名}可在正文中引用
\begin{definition}[定义名]\label{def1}
定义内容.
\end{definition}



%%% 单图
%%% 可在文中使用图\ref{fig1}引用图编号
\begin{figure}[!t]
\centering
\includegraphics{fig1.eps}
\cnenfigcaption{中文图题}{Caption}
\label{fig1}
\end{figure}

%%% 并排图
%%% 可在文中使用图\ref{fig1}、图\ref{fig2}引用图编号
\begin{figure}[!t]
\centering
\begin{minipage}[c]{0.48\textwidth}
\centering
\includegraphics{fig1.eps}
\end{minipage}
\hspace{0.02\textwidth}
\begin{minipage}[c]{0.48\textwidth}
\centering
\includegraphics{fig2.eps}
\end{minipage}\\[3mm]
\begin{minipage}[t]{0.48\textwidth}
\centering
\cnenfigcaption{中文图题1}{Caption1}
\label{fig1}
\end{minipage}
\hspace{0.02\textwidth}
\begin{minipage}[t]{0.48\textwidth}
\centering
\cnenfigcaption{中文图题2}{Caption2}
\label{fig2}
\end{minipage}
\end{figure}

%%% 并排子图
%%% 需要英文分图题 (a)...; (b)...
\begin{figure}[!t]
\centering
\begin{minipage}[c]{0.48\textwidth}
\centering
\includegraphics{subfig1.eps}
\end{minipage}
\hspace{0.02\textwidth}
\begin{minipage}[c]{0.48\textwidth}
\centering
\includegraphics{subfig2.eps}
\end{minipage}
\cnenfigcaption{中文图题}{Caption}
\label{fig1}
\end{figure}

%%% 算法
%%% 可在文中使用 算法\ref{alg1} 引用算法编号
\begin{algorithm}
%\floatname{algorithm}{Algorithm}%更改算法前缀名称
%\renewcommand{\algorithmicrequire}{\textbf{Input:}}% 更改输入名称
%\renewcommand{\algorithmicensure}{\textbf{Output:}}% 更改输出名称
\footnotesize
\caption{算法标题}
\label{alg1}
\begin{algorithmic}[1]
    \REQUIRE $n \geq 0 \vee x \neq 0$;
    \ENSURE $y = x^n$;
    \STATE $y \Leftarrow 1$;
    \IF{$n < 0$}
        \STATE $X \Leftarrow 1 / x$;
        \STATE $N \Leftarrow -n$;
    \ELSE
        \STATE $X \Leftarrow x$;
        \STATE $N \Leftarrow n$;
    \ENDIF
    \WHILE{$N \neq 0$}
        \IF{$N$ is even}
            \STATE $X \Leftarrow X \times X$;
            \STATE $N \Leftarrow N / 2$;
        \ELSE[$N$ is odd]
            \STATE $y \Leftarrow y \times X$;
            \STATE $N \Leftarrow N - 1$;
        \ENDIF
    \ENDWHILE
\end{algorithmic}
\end{algorithm}

%%% 简单表格
%%% 可在文中使用 表\ref{tab1} 引用表编号
\begin{table}[!t]
\cnentablecaption{表题}{Caption}
\label{tab1}
\footnotesize
\tabcolsep 49pt %space between two columns. 用于调整列间距
\begin{tabular*}{\textwidth}{cccc}
\toprule
  Title a & Title b & Title c & Title d \\\hline
  Aaa & Bbb & Ccc & Ddd\\
  Aaa & Bbb & Ccc & Ddd\\
  Aaa & Bbb & Ccc & Ddd\\
\bottomrule
\end{tabular*}
\end{table}

%%% 换行表格
\begin{table}[!t]
\cnentablecaption{表题}{Caption}
\label{tab1}
\footnotesize
\def\tabblank{\hspace*{10mm}} %blank leaving of both side of the table. 左右两边的留白
\begin{tabularx}{\textwidth} %using p{?mm} to define the width of a column. 用p{?mm}控制列宽
{@{\tabblank}@{\extracolsep{\fill}}cccp{100mm}@{\tabblank}}
\toprule
  Title a & Title b & Title c & Title d \\\hline
  Aaa & Bbb & Ccc & Ddd ddd ddd ddd.

  Ddd ddd ddd ddd ddd ddd ddd ddd ddd ddd ddd ddd ddd ddd ddd ddd ddd ddd ddd ddd ddd ddd ddd ddd ddd ddd ddd ddd ddd ddd ddd.\\
  Aaa & Bbb & Ccc & Ddd ddd ddd ddd.\\
  Aaa & Bbb & Ccc & Ddd ddd ddd ddd.\\
\bottomrule
\end{tabularx}
\end{table}

%%% 单行公式
%%% 可在文中使用 (\ref{eq1})式 引用公式编号
%%% 如果是句子开头, 使用 公式(\ref{eq1}) 引用
\begin{equation}
A(d,f)=d^{l}a^{d}(f),
\label{eq1}
\end{equation}

%%% 不编号的单行公式
\begin{equation}
\nonumber
A(d,f)=d^{l}a^{d}(f),
\end{equation}

%%% 公式组
\begin{eqnarray}
\nonumber
&X=[x_{11},x_{12},\ldots,x_{ij},\ldots ,x_{n-1,n}]^{\rm T},\\
\nonumber
&\varepsilon=[e_{11},e_{12},\ldots ,e_{ij},\ldots ,e_{n-1,n}],\\
\nonumber
&T=[t_{11},t_{12},\ldots ,t_{ij},\ldots ,t_{n-1,n}].
\end{eqnarray}

%%% 条件公式
\begin{eqnarray}
\sum_{j=1}^{n}x_{ij}-\sum_{k=1}^{n}x_{ki}=
\left\{
\begin{aligned}
1,&\quad i=1,\\
0,&\quad i=2,\ldots ,n-1,\\
-1,&\quad i=n.
\end{aligned}
\right.
\label{eq1}
\end{eqnarray}

%%% 其他格式
\footnote{Comments.} %footnote. 脚注
\raisebox{-1pt}[0mm][0mm]{xxxx} %put xxxx upper or lower. 控制xxxx的垂直位置

%%% 图说撑满
\Caption\protect\linebreak \leftline{Caption}
